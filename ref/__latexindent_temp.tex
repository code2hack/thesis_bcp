%Chap1绪论
@misc{NetAppFas,
Howpublished = {\url{https://www.netapp.com/us/documentation/fas-storage-systems.aspx}},
Title = {NetApp FAS Storage Systems}
}
%是一个开源、可扩展的分布式文件系统
@misc{GlusterFS,
	Howpublished = {\url{http://www.gluster.org/}},
	Title = {GlusterFS: A scalable network filesystem}
  }

%-----Chap2

%层次存储
@INPROCEEDINGS{Adaptive_Data_Migration_in_Multi-tiered_Storage_Based_Cloud_Environment,
author={G. {Zhang} and L. {Chiu} and L. {Liu}},
booktitle={2010 IEEE 3rd International Conference on Cloud Computing},
title={Adaptive Data Migration in Multi-tiered Storage Based Cloud Environment},
year={2010},
volume={},
number={},
pages={148-155},
keywords={data handling;disc storage;Internet;storage management;adaptive data migration;multitiered storage based cloud environment;solid state disks;SSD technology;hard disks;look ahead data migration model;Time factors;Adaptation model;Data models;Optimization;Heating;Computational modeling;Resource management;Cloud;storage system;solid state disk;data migration},
doi={10.1109/CLOUD.2010.60},
ISSN={},
month={July},}
@Article{A_Reinforcement_Learning_Framework_for_Online_data_migration,
author="Vengerov, David",
title="A reinforcement learning framework for online data migration in hierarchical storage systems",
journal="The Journal of Supercomputing",
year="2008",
month="Jan",
day="01",
volume="43",
number="1",
pages="1--19",
abstract="Multi-tier storage systems are becoming more and more widespread in the industry. They have more tunable parameters and built-in policies than traditional storage systems, and an adequate configuration of these parameters and policies is crucial for achieving high performance. A very important performance indicator for such systems is the response time of the file I/O requests. The response time can be minimized if the most frequently accessed (``hot'') files are located in the fastest storage tiers. Unfortunately, it is impossible to know a priori which files are going to be hot, especially because the file access patterns change over time. This paper presents a policy-based framework for dynamically deciding which files need to be upgraded and which files need to be downgraded based on their recent access pattern and on the system's current state. The paper also presents a reinforcement learning (RL) algorithm for automatically tuning the file migration policies in order to minimize the average request response time. A multi-tier storage system simulator was used to evaluate the migration policies tuned by RL, and such policies were shown to achieve a significant performance improvement over the best hand-crafted policies found for this domain.",
issn="1573-0484",
doi="10.1007/s11227-007-0135-3",
url="https://doi.org/10.1007/s11227-007-0135-3"
}


%是操作系统数据读取过程中的重要优化方法
@inproceedings{Reducing_File_System_Latency_using_a_Predictive_Approach,
  title={Reducing File System Latency using a Predictive Approach.},
  author={Griffioen, Jim and Appleton, Randy},
  booktitle={USENIX summer},
  pages={197--207},
  year={1994}
} 
@inproceedings{Group_based_management_of_distributed_file_caches,
  title={Group-based management of distributed file caches},
  author={Amer, Ahmed and Long, Darrell DE and Burns, Randal C},
  booktitle={Proceedings 22nd International Conference on Distributed Computing Systems},
  pages={525--534},
  year={2002},
  organization={IEEE}
}
@article{A_data_mining_algorithm_for_generalized_web_prefetching,
  title={A data mining algorithm for generalized web prefetching},
  author={Nanopoulos, Alexandros and Katsaros, Dimitrios and Manolopoulos, Yannis},
  journal={IEEE Transactions on Knowledge \&amp; Data Engineering},
  number={5},
  pages={1155--1169},
  year={2003},
  publisher={IEEE}
}